\section{Fazit} \label{sec:Fazit}
In dieser arbeit konnte natürlich ein eine endliche Anzahl an Ansätzen und Ideen untersucht werden. In Zukunft ist bestimmt noch viel kreativität und Forschungspotential in diesem Bereich. Außerdem vielleicht sogar komplett neue Ansätze, die nicht auf der Stadt-Metapher basieren sondern vielleicht auf planeten oder anderen Metaphern, auch wenn zu vermuten ist, dass dabei die übersichlichekeit und einfachheit leiden könnte. Die in diser arbeit untersuchten Ansätze sind alle sehr einfach mit basic formen und daher übersichlich.


Es ist natürlich klar, dass der treempa ansatz optmiert wurde und die anderen Ansätze nicht, weshalb der treemap ansatz besere ergebnisse liefert. 
Es bleibt zu erforschen, ob sich die anderen Ansätze auch so optimieren lassen, dass sie bessere Ergebnisse liefern.


Außerdem ist bei der literatur recherche noch interessant auch komplett über den tellerrand von software hinaus zu schauen und andere hierarchical vis zu betrachten, die nicht direkt mit software zu tun haben, um eventuell neue gut viz zu finden. Oder sogar ganz anders mit Kreativen ideen findungs ansätzen komplett neue Ansätze zu finden. Oder sich neue methaphern zu überlegen und nicht nur stadt, inseln, weltall,....

Außerdem ist es interannt in der umfrage die tools den nutzern direkt zur verfügung zu stellen. In dieser arbeit wurden tools nur mit statischen bildern vorgestllt, um sich konkret nur auf die Visualisierung zu konzentrieren und faktoren wie nutzerfreundlichkeit, performance und interaktion nicht mit zu betrachten.

"Außerdem kann es passieren, dass die Beschriftung zu lang für die breite der Knotens ist, wodurch die Beschriftung entweder abgeschnitten wird oder über den Knoten hinauswächst." das angehen, mit zum beispiel rechteck breiter machen oder so, anders paltzieren, PIPAPO

Außerdem eventuell dynamische Beschriftung implementieren