\section{Grundlagen} \label{sec:Grundlagen}


\subsection{Treemap-Layouts} \label{sec:Treemap}

Eine Treemap visualisiert einen Baum, indem jedem Knoten ein Rechteck mit der Fläche A zugewiesen wird, proportional zu seinem zugewiesenen Wert (z.B. Datenmenge oder Marktwert). Nicht-Blatt-Knoten werden dabei üblicherweise durch Rahmen (Container-Rectangles) gekennzeichnet, um die Gruppierung der Kinder zu zeigen. \cite{bruls2000squarified} Die Rechtecke aller Blätter füllen die Fläche des Wurzelrechtecks vollständig aus. Mathematisch entspricht die Eingabedatenstruktur einem gewichteten Baum, bei dem jede Blatteinheit eine numerische Größe hat. Die Fläche eines Eltern-Rechtecks entspricht der Summe der Flächen (Werte) seiner Kinder.



\subsubsection{Treemap-Algorithmen} \label{sec:TreemapAlgo}

\subsubsection{Squarify-Algorithmus} \label{sec:Squarify}



