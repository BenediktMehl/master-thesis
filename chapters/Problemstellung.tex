\section{Problemstellung} \label{sec:Problemstellung}


\subsection{Code Qualität greifbar machen} \label{sec:CodeQualitaetGreifbar}
Es gibt bereits viele ansätze code zu visualisieren und es gibt auch viele Ansätze dies drei dimensional zu tun, um die greifbarkeit und die plastizität zu erhöhen, um so das Verständnis zu fördern \cite{3dsoftwareMarcus,codeCity1,first_3D_vis,virtualButVisibleMunro}. Spezieller wird, es wenn es darum geht die Qualität des codes 3d zu visualisieren. Am bekanntesten und am erfolgreichsten ist hier die Code-Stadt Analogie, die in Abschnitt \ref{sec:CodeCity} beschrieben wurde. Die fundamentale und schwerste Frage, bei den Stadt-Analogien ist, wie das Layout der Stadt aussehen soll.

\subsection{Das Treemap Problem} \label{sec:TreemapProblem}
In Abschnnit \ref{sec:Treemap} wurde aufgezeigt, dass bereits der initiale Algorithmus von Johnson und Shneiderman \cite{johnson1991tree} ein fundamentales Problem aufweist, wenn Treemaps mit Abständen zwischen Knoten dargestellt werden sollen. 
\begin{itemize}
    \item Da der Abstand von der Fläche der Knoten abgezogen wird, ist die dargestellte Fläche nicht mehr proportional zum Wert des Knotens.
    \item Durch das Abziehen der Abstände kann es passieren, dass Knoten verschwinden, wenn entweder die Länge oder die Breite der Knoten kleiner oder gleich dem Abstand ist.
\end{itemize}


Warum stadt? es ist einfach und mit 3d blöcken anders als zb. wald metaphern\cite{softwareForest}