\section{Problemstellung} \label{sec:Problemstellung}

Obwohl bereits leistungsfähige Werkzeuge zur Visualisierung existieren, stehen wir vor
zentralen Herausforderungen bei der Darstellung von Codequalitätsmetriken:
• Aussagekraft: Wie lässt sich die Aussagekraft einer Visualisierung erhöhen? Wie
viele Metriken können simultan dargestellt werden?
• Übersichtlichkeit: Wie lässt sich die Übersichtlichkeit der Darstellung erhöhen?
Ermöglicht sie eine gezielte Erkennung individueller Qualitätsprobleme? Ist eine
schnelle Erfassung der Inhalte ohne umfangreiches Vorwissen und Zeitaufwand
möglich?
• Skalierbarkeit: Wie lassen sich umfangreiche Codebasen effektiv visualisieren,
ohne an Übersichtlichkeit einzubüßen? Wie schnell können diese
Visualisierungen generiert werden?

Zielsetzung
Das Ziel dieser Arbeit ist es, Anforderungen an eine effektive Visualisierung von
Codequalitätsmetriken zu definieren und zu zeigen, ob eine verbesserte
Darstellungsmethode existiert.



----------
Was ist uns maibornwolff wichtig bei dem layout?


Christian Hühn gespräch:
Eine Sortierung innerhalb der Ordner ist nicht so wichtig, da die Sortierung eigentlich schon über die ordner struktur hergestellt wird. Außerdem nach was, es macht das bild nicht wirklich übersichtlicher

Kohäsion: entweder gut strukturiert oder schlecht - dann bringt das eh nichts



Ziel ist es schnell und einfach Hotspots zu finden:
Code Charta ist die ausgangslage um die größten Hotspots zu finden

Geredet wird meistens mit Entwicklern

CodeCharta hilft systeme schnell zu verstehen und wo man hinschauen muss
Schnell einen Überblick verschaffen und CodeBasis greifbar machen

Durch 3D mehr verbindungen herstellen: Zb sind große Gebäude auch gut gecovert



Software entwickler nutzen für die analyse selbst erstmal 
Selbst entwickler geben gutes Feedback


Streemap ist nicht so intuitiv - Vorteile sind, dass es sehr stabil ist




Kunden zeigen wie sich ein Projekt geändert hat:
Bei modernisierungs Projekten 
Ansonsten bleibt in audits die basisfläche gleich, somit ist das nicht so wichtig

DD-Audits: 
Screenshots von ein paar Ansichten
Sprechen über hotspots

Software Health-Check
Kunde hat system, was ist da nicht rund?

Code Quality Insights
Screenshots generieren


----

moment mal: ich will eigentlich nur dass die kind knoten ein gutes Seitenverhältnis haben, der rest kann mir ja eigentlich egal sein - oder zumindest nicht so wichtig


\subsection{Das Treemap Problem} \label{sec:TreemapProblem}
In Abschnnit \ref{sec:Treemap} wurde aufgezeigt, dass bereits der initiale Algorithmus von Johnson und Shneiderman \cite{johnson1991tree} ein fundamentales Problem aufweist, wenn Treemaps mit Abständen zwischen Knoten dargestellt werden sollen. 
\begin{itemize}
    \item Da der Abstand von der Fläche der Knoten abgezogen wird, ist die dargestellte Fläche nicht mehr proportional zum Wert des Knotens.
    \item Durch das Abziehen der Abstände kann es passieren, dass Knoten verschwinden, wenn entweder die Länge oder die Breite der Knoten kleiner oder gleich dem Abstand ist.
\end{itemize}

