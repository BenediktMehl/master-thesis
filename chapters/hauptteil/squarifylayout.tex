\section{Stadt Layout Anpassungen} \label{sec:SquarifyLayoutAnpassungen}

In diesem Abschnitt geht es nicht um die Algorithmen, die zur erzeugung der Layouts verwendet werden, sondern es geht um die Layouts selbst. Hier wird immer konkret begründed, warum ich mich für bestimmte layouts entschieden habe und eventuell auch warum ich mich gegen andere entschieden habe. Dann werden diese Layouts - auf das in dieser Arbeit behandelte konkrete Problem - angepasst, damit sie dann am Ende evaliuiert werden können.  

CodeCity:
Optionen:
Eins:
Die erste Änderung ist, dass die Knoten nicht mehr quadratisch sind, um den Platz besser auszunutzen und weniger freie ungenutzte Flächen zu haben.
Zwei:
Abstände zwischen Knoten könnten auf top level höher sein und nach unten hin immer kleiner werden, um den Platz besser auszunutzen und Probleme mit den Algorithmen zu verkleinern.
Drei:
Ordner könnten ein Label bekommen, dass die Fläche in eine dimension vergrößert. Anonsten wird das schwer zb. mit schwebenden Labels, da wird alles unübersichtlich. Das vielleicht auch nur auf den Top Level Ordnern?



Ganz Andere Idee:
Street maps???
also order werden zu straßen und die Knoten zu Häusern



Ganz Andere Idee:
Kreis förmig anordnen? Es gibt ja auch runde städte
(oder bar chart)




Ganz Andere Idee:
Circle packing

