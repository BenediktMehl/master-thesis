\section{Datenanalyse} \label{sec:datenAnalyse}

Im Grundlagen Kapitel wurde bereits erwähnt, dass die Metrik-Daten, die wir durch die Analyse von Code erhalten, spezielle Eigenschaften aufweist (siehe Abschnitt \ref{sec:SoftwareQualitaetsmetriken}). Diese Eigenschaften sind wichtig, um die Visualierung zu optimieren und zu wissen in welchem Rahmen sich die zu visualisierenden Daten bewegen. 
In diesem Abschnitt werden wir verschiedene Open-Source-Projekte analysieren und die Metrik-Daten untersuchen, um ein besseres Verständnis für die Struktur von Softwareprojekten zu gewinnen. 
Wir sind uns bewusst, dass ein gewisser bias bestehen könnte, da open source projekte eventuell nicht die gleiche Struktur wie kommerzielle Softwareprojekte aufweisen. Dennoch glauben wir, dass ein allgemeiner Trend erkennbar sein wird, der uns bei der Visualisierung helfen kann. QUELLE für diese Annahme?

Als Quelle werden wir GitHub verwenden, da es eine große Anzahl an Open-Source-Projekten bietet und die Daten leicht zugänglich sind.
Wir probieren möglich breit gefächert Projekte zu betrachten, um ein möglichst breites Spektrum an Softwareprojekten abzudecken. 
Das Ziel bei der Auswahl der Projekte ist es, eine möglichst diverse Auswahl an Projekten zu haben, um später eine Aussage über die Struktur von Softwareprojekten im Allgemeinen treffen zu können und nicht nur über die Stuktur der meisten Softwareprojekte. Wir könnten einfach zufällige Projekte von Github auswählen, aber das würde wahrscheinlich zu einer Verzerrung führen und eher kleinere und unbekannte Projekte auswählen.
Deshalb setzen wir fünf Kriterien für die Auswahl der Projekte fest, die wir im Folgenden näher erläutern werden.

\begin{itemize}
    \item \textbf{Popularität:} Wir wollen Projekte mit einer verschieden großen Popularität betrachten, zwar deutet eine Studie an, dass Popularität nicht in Korrelation mit Software Qualität steht \cite{popAndQuality}. Außerdem sind die meisten Projekte die auf GitHub viele Sterne haben eher Sammlungen von Bücher, Kursen oder anderen Ressourcen und nicht wirklich Softwareprojekte \cite{evanli/github-ranking_2025}.
    trotzdem haben Populärere Projekte oft eine größere Anzahl an Mitwirkenden, mehr Änderungen und generell mehr aufmerksamkeit. 
    unterschiedlichen Anzahlen an Mitwirkenden hat einen Einfluss auf die Struktur und Qualität des Codes haben kann \cite{numDevs}. Wir werden also die Popularität nicht anhand der Anzahl der Sterne messen, sondern anhand der Anzahl der Mitwirkenden. Wir werden nur Projekte bereits ab einer Anzahl von 2 Mitwirkenden betrachten, da Projekte mit nur einem Mitwirkenden oft nicht wirklich repräsentativ für die Struktur von Softwareprojekten sind. Außerdem hat unsere analyse von 1200 Github Projekten gezeigt, dass keine (ohne nur eine sehr schwache positive) Korrelation zwischen der Anzahl der Sterne und der Anzahl der Mitwirkenden besteht (Wert von 0.15).
    \item \textbf{Programmiersprache:} Wir werden Projekte in verschiedenen Programmiersprachen betrachten, da die Struktur von Softwareprojekten in verschiedenen Programmiersprachen stark unterschiedlich sein kann. Wir wählen die Programmiersprachen nach ihrer Popularität und Paradigma aus. 
    \begin{itemize}
        \item \textbf{Python:} Die zweit populärste Sprache \cite{software_state_2022} mit dynamischer Typisierung, sowohl imperativem als auch objektorientiertem Paradigma. 
        \item \textbf{JavaScript:} die populärste Sprache \cite{software_state_2022} mit dynamischer Typisierung. Die Struktur von JavaScript-Projekten ist besonders geprägt durch Frameworks.
        \item \textbf{Java:} Die dritt populärste Sprache \cite{software_state_2022} mit statischer Typisierung und objektorientiertem Paradigma. Java-Projekte sind oft sehr strukturiert und folgen bestimmten Konventionen.
        \item \textbf{C:} Eine der ältesten Programmiersprachen, die immer noch weit verbreitet ist (Platz 9 \cite{software_state_2022}). C-Projekte sind oft sehr nah an der Hardware und haben eine andere Struktur als Projekte in höheren Programmiersprachen.
        \item \textbf{Shell:} Eine Skriptsprache (Platz 8 \cite{software_state_2022}), die oft für Automatisierung und Systemadministration verwendet wird.
    \end{itemize}
    \item \textbf{Startdatum und Aktivität} Programmierstiele und -konventionen ändern sich im Laufe der Zeit. Da unsere Visualisierung sowohl für alte legacy-Projekte als auch für neue Projekte geeignet sein soll, werden wir Projekte aus verschiedenen Zeitperioden betrachten. Das selbe gilt für die Aktivität des Projekts. Wir messen das Startdatum als das Datum, an dem das Repository erstellt wurde, und die Aktivität als die letze Änderung um code - der letzte Push - in das Repository. Das älteste Projekt, das wir betrachten, ist von 2008 und das Projekt mit der frühesten letzen Änderung ist von 2013.
    \item \textbf{Projektgröße:} Die Projektgröße ist offensichlich der Wichtigste Faktor für die STruktur eines Projekts. Große Projekte haben tiefere Hierarchien und generell mehr Dateien. Wir messen die Projekt größe als die Summe der Dateigröße aller dateien in dem REpo. Wir haben uns dazu entschieden nur Projekte mit einer mindest Größe von 100 KB zu betrachten, da kleinere Projekte oft einfach keinen Sinn machen zu visualisieren, da sie auch so schon gut zu überschauen sind.
\end{itemize}
    


