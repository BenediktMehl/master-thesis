\section{Datenanalyse} \label{sec:datenAnalyse}

Im Grundlagen Kapitel wurde bereits erwähnt, dass die Metrik-Daten, die wir durch die Analyse von Code erhalten, spezielle Eigenschaften aufweist (siehe Abschnitt \ref{sec:SoftwareQualitaetsmetriken}). Diese Eigenschaften sind wichtig, um die Visualierung zu optimieren und zu wissen in welchem Rahmen sich die zu visualisierenden Daten bewegen. 
In diesem Abschnitt werden wir verschiedene Open-Source-Projekte analysieren und die Metrik-Daten untersuchen, um ein besseres Verständnis für die Struktur von Softwareprojekten zu gewinnen. 
Wir sind uns bewusst, dass ein gewisser bias bestehen könnte, da open source projekte eventuell nicht die gleiche Struktur wie kommerzielle Softwareprojekte aufweisen. Dennoch glauben wir, dass ein allgemeiner Trend erkennbar sein wird, der uns bei der Visualisierung helfen kann. QUELLE für diese Annahme?

Wir probieren möglich breit gefächert Projekte zu betrachten, um ein möglichst breites Spektrum an Softwareprojekten abzudecken. 
Kriterien für die Auswahl der Projekte:
\begin{itemize}
    \item \textbf{Programmiersprache:} Wir wollen Projekte in verschiedenen Programmiersprachen betrachten, um zu sehen, ob es Unterschiede in der Struktur gibt.
    \item \textbf{Projektgröße:} Wir wollen sowohl kleine als auch große Projekte betrachten, um zu sehen, ob es Unterschiede in der Struktur gibt.
    \item \textbf{Projektart:} Wir wollen verschiedene Arten von Projekten betrachten, z.B. Bibliotheken, Frameworks, Anwendungen, um zu sehen, ob es Unterschiede in der Struktur gibt.
    \item \textbf{Aktivität:} Wir wollen sowohl aktive als auch inaktive Projekte betrachten, um zu sehen, ob es Unterschiede in der Struktur gibt.
    \item \textbf{Popularität:} Wir wollen sowohl populäre als auch weniger bekannte Projekte betrachten, um zu sehen, ob es Unterschiede in der Struktur gibt.
    \item \textbf{Alter:} Wir wollen sowohl neue als auch alte Projekte betrachten, um zu sehen, ob es Unterschiede in der Struktur gibt.
\end{itemize}
Als Quelle werden wir GitHub verwenden, da es eine große Anzahl an Open-Source-Projekten bietet und die Daten leicht zugänglich sind.
