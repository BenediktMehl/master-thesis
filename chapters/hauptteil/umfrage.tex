Interessante Ideen von anderen papern zur umfragen gestaltung:

Anti patterns finden
Anti-patterns are known to be bad
coding practices that may cause problems in subsequent development phases. An example of anti-pattern detectable by
our framework is the Blob [3].
\cite{visbasedlarge}
In this experiment we use three layout techniques: Treemap,
Sunburst, and for comparison purposes, a na¨ıve layout technique called Treeline.
Treeline represents classes of the architecture hierarchy
with a depth-first algorithm. Each time we meet a node in
the architecture tree, we place it on the current row, distributing the elements from left to right. Levels and packages are determined with separators, with a given separator color for each level of the hierarchy. So when the color
switches, the following classes are in a different level. When
the separator color remains the same, the next package is a
sibling. This algorithm has an optimal use of the space and
is easy to implement/execute. An example of this layout
technique is presented in Figure 12.
The experiments are run in the form of an electronic questionnaire with 20 analysis tasks. Two types of knowledge are
needed to perform the tasks: class characteristics and program architecture. 5 tasks involve exclusively the first type
of knowledge, 5 tasks involve the second type, and 10 tasks
involve both types. Task definition is inspired by the types
we described in Section 5. For example, one task is to identify large packages that contain almost exclusively highly
cohesive classes. Each task had to be performed on a different program, taken from different application domains, and
with sizes ranging from 72 to 1662 classes. For instance, the
Figure 12: Example of the Treeline algorithm. It
represents EMMA, a tool for measuring coverage of
Java software (286 classes).
task given as training used the associated program JDK 1.1
(1662 classes).
The time taken by a subject to correctly perform the task
is automatically recorded. In addition to the 20 tasks, subjects were asked questions about their subjective rating of
the layout techniques. We used 15 subjects (graduate students) divided into 3 groups of 5 subjects. For each analysis
task, subjects of each group are asked to perform the task
using one of the three layout techniques (Treemap for group
A, Sunburst for group B, and Treeline for group C). To avoid
fatigue and learning effect biases, the assignation of a layout
technique to a group is random and changed with the tasks.
The order of tasks for subjects from the same group is also
random. For each pair (task:layout), we computed the average time of the 5 subjects. We used the computed value
to rank the techniques for each task. Then, for each layout
technique, we calculated the average and the median for all
tasks.
Subjects are volunteers. Their motivation should not be
biased by any form of evaluation. Most of them are software
engineering researchers. To avoid significant differences between them, they received a quick training on the environment before the experiment. They learned how classes and
221
packages are represented within each layout technique, how
to navigate in this 3D space, and finally how to perform the
tasks and record time.