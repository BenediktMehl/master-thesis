\section{Motivation} \label{sec:Motivation}
Die Qualität von Software für Menschen verständlich und greifbar zu machen, die nicht täglich mit Quellcode arbeiten, stellt nach wie vor eine Herausforderung dar. Aber auch erfahrene Entwickler*innen stehen regelmäßig vor der Aufgabe, sich schnell einen Überblick über ein bestehendes Softwaresystem zu verschaffen: Wo liegen die problematischen Stellen? Wo lohnt sich ein tieferer Blick? Und wie lassen sich Verbesserungspotenziale effizient identifizieren?

Um solche Fragen beantworten zu können, sind geeignete Visualisierungen essenziell. Sie helfen, komplexe Strukturen zu abstrahieren, Muster zu erkennen und technische Schulden sichtbar zu machen, ohne dass zunächst jede Zeile Code gelesen werden muss. Während einfache Diagramme oder Dashboards nützliche Einstiegspunkte bieten, zeigen immersive Ansätze, insbesondere dreidimensionale Darstellungen, ein deutlich höheres Potenzial, Software auf eine intuitivere und erfahrbare Weise zugänglich zu machen \cite{3dsoftwareMarcus,codeCity1,first_3D_vis,virtualButVisibleMunro}.

Insbesondere 3D-Metaphern, wie die Stadt-Metapher \cite{codeCity1}, haben sich in Forschung und Praxis als äußerst wirkungsvoll erwiesen (QUELLEN). Durch die Übertragung von Softwarestrukturen auf städtische Elemente (Gebäude, Blöcke, Straßen) entsteht ein räumliches Abbild, das verschiedene Code-Metriken kombinieren kann. Die Beliebtheit dieser Methode beruht nicht zuletzt auf der hohen Informationsdichte und der intuitiven Lesbarkeit räumlicher Strukturen. Im Vergleich dazu wirken alternative Metaphern, etwa waldartige Darstellungen \cite{softwareForest}, häufig unübersichtlich und verlieren bei wachsender Komplexität schnell an Aussagekraft.

Gleichzeitig zeigt sich, dass auch die Stadt-Metapher nicht frei von Schwächen ist. Gerade bei sehr großen Codebasen stößt die Übersichtlichkeit an ihre Grenzen oder wichtige Details gehen verloren \cite{lu2008cascaded}. Viele dieser Visualisierungen beruhen auf Treemap-Layouts, einer etablierten Technik zur Darstellung hierarchischer Strukturen. Doch gerade in Bezug auf Lesbarkeit, Informationsdichte und Benutzerfreundlichkeit stellt sich zunehmend die Frage, ob der klassische Treemap-Ansatz nicht in Bezug auf 3D-Softwarevisualisierungen verbessert werden kann.
%Warum zunehmen, warum jetzt immer mehr? In dieser vllt schon vorgreifen, auf was ich mich hier beziehe

Außerdem ist offen, ob es alternative Layout-Ansätze gibt, die eine noch klarere, verständlichere oder flexiblere Darstellung sowohl in 2D als auch in 3D ermöglichen würden. In der Praxis ist dieser Aspekt bisher wenig beleuchtet worden, obwohl er entscheidend für die Nützlichkeit solcher Visualisierungen ist.

