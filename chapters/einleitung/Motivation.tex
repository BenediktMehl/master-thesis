\section{Motivation} \label{sec:Motivation}
Es stellt sich immer die Frage, wie man Software qualität für leute greifbar machen kann, die nicht unbedingt Software-Entwickler sind. Außerdem wie kann auf einen blick als Software-Entwickler die Qualität des Codes einschätzen, ohne den Code gelesen zu haben bzw wie kann ich einen möglichst einfachen und schnellen einstieg in Code bekommen, um zu verstehen, wo ich anfangen muss zu verbessern.
Es gibt unzählige Möglichkeiten, Software-Qualität zu visualisieren. Sei es in einfachen Diagrammen oder komplexen interaktiven Dashboards. 
Es gibt bereits viele ansätze code zu visualisieren und es gibt auch viele Ansätze dies drei dimensional zu tun, um die greifbarkeit und die plastizität zu erhöhen, um so das Verständnis zu fördern \cite{3dsoftwareMarcus,codeCity1,first_3D_vis,virtualButVisibleMunro}. 
Der Vorteil von 3D Visualisierungen ist, dass sie eine wirklich räumliche Vorstellung des Codes Ermöglichen und eine beinahe immersive Erfahrung bieten. Dies kann helfen, Muster und Strukturen im Code zu erkennen, die in 2D-Darstellungen nicht abgebildet werden können. Außerdem lassen sich durch die Zusätzliche Dimensionen mehr Informationen in einem einzigen Bild darstellen.
Weit verbreitet ist die Nutzung von 3D-STadt-Methaphern \cite{codeCity1}, um Software-Qualität zu visualisieren. Diese Methoden haben aber aktuell immernoch einige Probleme, speziell, wenn es um die Übersichtlichkeit geht \cite{lu2008cascaded}. Viele dieser Ansätze nutzen Treemaps-Layout und Algorithmen. Die Frage ist ob herkömmliche Treemaps-Algorithmen überhaupt geeignet sind, um wirklich übersichliche Treemap-Layouts zu erzeugen, oder ob nicht andere Ansätze besser geeignet sind. Dies wurde bisher noch nicht ausreichend untersucht.




Warum stadt? es ist einfach und mit 3d blöcken anders als zb. wald metaphern\cite{softwareForest}