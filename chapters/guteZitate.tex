\enquote{Treemaps are very effective when size is the most important feature to be displayed.}\cite[2]{bruls2000squarified}


\enquote{[node and link] diagrams are very effective for small trees, but usually fall short when more than a couple of hundred elements have to be visualized simultaneously.}\cite[2]{bruls2000squarified}

\enquote{Human brains are able to process certain visual information preattentively: attributes such as length can be grasped quickly and accurately, and values for such attributes can be compared with almost no cognitive effort. Unfortunately, area is not one of these preattentive attributes. Treemaps rely on area (and possibly color) to encode the value of a variable, and therefore, although treemaps can convey overall relationships in a large data set, they are not suited for tasks involving precise comparisons.}\cite{laubheimer_2019}

\enquote{As humans we have the ability to recognize the spatial configuration of
elements in a picture and notice the relationships between elements quickly}\cite[2]{johnson1998tree}

\enquote{This is in contrast to traditional static methods of displaying
hierarchically structured information, which generally make either poor use of
display space or hide vast quantities of information from users.}\cite[2]{johnson1998tree}

\enquote{Displaying a directory tree while fully utilizing space and conveying structural information in a visually appealing and low cognitive load manner is a difficult task, as these are often opposing goals.}\cite[8]{johnson1998tree}

\enquote{The research being described in this paper has tried to harness natural perceptual skills in this way, through the use of a very strong real world metaphor.}\cite[2]{virtualButVisibleMunro}

