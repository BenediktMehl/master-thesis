% ****************************************************************************************************
% hdathesis-config.tex 
% Use it at the beginning of your thesis.tex, or as a LaTeX Preamble 
% in your thesis.{tex,lyx} with % ****************************************************************************************************
% hdathesis-config.tex 
% Use it at the beginning of your thesis.tex, or as a LaTeX Preamble 
% in your thesis.{tex,lyx} with % ****************************************************************************************************
% hdathesis-config.tex 
% Use it at the beginning of your thesis.tex, or as a LaTeX Preamble 
% in your thesis.{tex,lyx} with % ****************************************************************************************************
% hdathesis-config.tex 
% Use it at the beginning of your thesis.tex, or as a LaTeX Preamble 
% in your thesis.{tex,lyx} with \input{hdathesis-config}
% ****************************************************************************************************

% ****************************************************************************************************
% 1. Personal data and user ad-hoc commands
% ****************************************************************************************************
\newcommand{\myTitle}{Optimierung von Softwarequalitäts-Visualisierung\xspace}
%\newcommand{\mySubtitle}{An Homage to The Elements of Typographic Style\xspace}
\newcommand{\myDegree}{Master of Science (M.\,Sc.)\xspace}
\newcommand{\myName}{Benedikt Mehl\xspace}
\newcommand{\myId}{1121684\xspace}
\newcommand{\myProf}{Prof. Dr. Kai Renz\xspace}
%\newcommand{\myOtherProf}{Prof. Dr. Martin Girschick\xspace}
\newcommand{\myOtherProf}{--\xspace}
\newcommand{\mySupervisor}{Dr. Andreas Blunk\xspace}
\newcommand{\myFaculty}{Fachbereich Informatik\xspace}
\newcommand{\myUni}{Hochschule Darmstadt\xspace}
\newcommand{\myLocation}{Darmstadt\xspace}
\newcommand{\myTime}{\today\xspace}
\newcommand{\myVersion}{Version 0.1}

% ****************************************************************************************************
% 2. Is it a master thesis?
% ****************************************************************************************************
\PassOptionsToPackage{master}{hdathesis} % uncomment if this is a master thesis 

% ****************************************************************************************************
% 3. Does the thesis have a lock flag?
% ****************************************************************************************************
%\PassOptionsToPackage{lockflag}{hdathesis} % uncomment if this thesis has a lock flag 

% ****************************************************************************************************
% 4. Loading some handy packages
% ****************************************************************************************************
% ****************************************************************************************************
% Packages with options that might require adjustments
% ****************************************************************************************************

%\PassOptionsToPackage{ngerman,american}{babel}   % change this to your language(s)
% Spanish languages need extra options in order to work with this template
%\PassOptionsToPackage{spanish,es-lcroman}{babel}
\usepackage{babel}


% ****************************************************************************************************

% ****************************************************************************************************
% 1. Personal data and user ad-hoc commands
% ****************************************************************************************************
\newcommand{\myTitle}{Optimierung von Softwarequalitäts-Visualisierung\xspace}
%\newcommand{\mySubtitle}{An Homage to The Elements of Typographic Style\xspace}
\newcommand{\myDegree}{Master of Science (M.\,Sc.)\xspace}
\newcommand{\myName}{Benedikt Mehl\xspace}
\newcommand{\myId}{1121684\xspace}
\newcommand{\myProf}{Prof. Dr. Kai Renz\xspace}
%\newcommand{\myOtherProf}{Prof. Dr. Martin Girschick\xspace}
\newcommand{\myOtherProf}{--\xspace}
\newcommand{\mySupervisor}{Dr. Andreas Blunk\xspace}
\newcommand{\myFaculty}{Fachbereich Informatik\xspace}
\newcommand{\myUni}{Hochschule Darmstadt\xspace}
\newcommand{\myLocation}{Darmstadt\xspace}
\newcommand{\myTime}{\today\xspace}
\newcommand{\myVersion}{Version 0.1}

% ****************************************************************************************************
% 2. Is it a master thesis?
% ****************************************************************************************************
\PassOptionsToPackage{master}{hdathesis} % uncomment if this is a master thesis 

% ****************************************************************************************************
% 3. Does the thesis have a lock flag?
% ****************************************************************************************************
%\PassOptionsToPackage{lockflag}{hdathesis} % uncomment if this thesis has a lock flag 

% ****************************************************************************************************
% 4. Loading some handy packages
% ****************************************************************************************************
% ****************************************************************************************************
% Packages with options that might require adjustments
% ****************************************************************************************************

%\PassOptionsToPackage{ngerman,american}{babel}   % change this to your language(s)
% Spanish languages need extra options in order to work with this template
%\PassOptionsToPackage{spanish,es-lcroman}{babel}
\usepackage{babel}


% ****************************************************************************************************

% ****************************************************************************************************
% 1. Personal data and user ad-hoc commands
% ****************************************************************************************************
\newcommand{\myTitle}{Optimierung von Softwarequalitäts-Visualisierung\xspace}
%\newcommand{\mySubtitle}{An Homage to The Elements of Typographic Style\xspace}
\newcommand{\myDegree}{Master of Science (M.\,Sc.)\xspace}
\newcommand{\myName}{Benedikt Mehl\xspace}
\newcommand{\myId}{1121684\xspace}
\newcommand{\myProf}{Prof. Dr. Kai Renz\xspace}
%\newcommand{\myOtherProf}{Prof. Dr. Martin Girschick\xspace}
\newcommand{\myOtherProf}{--\xspace}
\newcommand{\mySupervisor}{Dr. Andreas Blunk\xspace}
\newcommand{\myFaculty}{Fachbereich Informatik\xspace}
\newcommand{\myUni}{Hochschule Darmstadt\xspace}
\newcommand{\myLocation}{Darmstadt\xspace}
\newcommand{\myTime}{\today\xspace}
\newcommand{\myVersion}{Version 0.1}

% ****************************************************************************************************
% 2. Is it a master thesis?
% ****************************************************************************************************
\PassOptionsToPackage{master}{hdathesis} % uncomment if this is a master thesis 

% ****************************************************************************************************
% 3. Does the thesis have a lock flag?
% ****************************************************************************************************
%\PassOptionsToPackage{lockflag}{hdathesis} % uncomment if this thesis has a lock flag 

% ****************************************************************************************************
% 4. Loading some handy packages
% ****************************************************************************************************
% ****************************************************************************************************
% Packages with options that might require adjustments
% ****************************************************************************************************

%\PassOptionsToPackage{ngerman,american}{babel}   % change this to your language(s)
% Spanish languages need extra options in order to work with this template
%\PassOptionsToPackage{spanish,es-lcroman}{babel}
\usepackage{babel}


% ****************************************************************************************************

% ****************************************************************************************************
% 1. Personal data and user ad-hoc commands
% ****************************************************************************************************
\newcommand{\myTitle}{Optimierung und Analyse dreidimensionaler Visualisierungstechniken für die effektive Darstellung von Software-Metriken\xspace}
%\newcommand{\mySubtitle}{An Homage to The Elements of Typographic Style\xspace}
\newcommand{\myDegree}{Master of Science (M.\,Sc.)\xspace}
\newcommand{\myName}{Benedikt Mehl\xspace}
\newcommand{\myId}{1121684\xspace}
\newcommand{\myProf}{Prof. Dr. Kai Renz\xspace}
%\newcommand{\myOtherProf}{Prof. Dr. Martin Girschick\xspace}
\newcommand{\myOtherProf}{--\xspace}
\newcommand{\mySupervisor}{Dr. Andreas Blunk\xspace}
\newcommand{\myFaculty}{Fachbereich Informatik\xspace}
\newcommand{\myUni}{Hochschule Darmstadt\xspace}
\newcommand{\myLocation}{Darmstadt\xspace}
\newcommand{\myTime}{\today\xspace}
\newcommand{\myVersion}{Version 0.1}

% ****************************************************************************************************
% 2. Is it a master thesis?
% ****************************************************************************************************
\PassOptionsToPackage{master}{hdathesis} % uncomment if this is a master thesis 

% ****************************************************************************************************
% 3. Does the thesis have a lock flag?
% ****************************************************************************************************
%\PassOptionsToPackage{lockflag}{hdathesis} % uncomment if this thesis has a lock flag 

% ****************************************************************************************************
% 4. Loading some handy packages
% ****************************************************************************************************
% ****************************************************************************************************
% Packages with options that might require adjustments
% ****************************************************************************************************

%\PassOptionsToPackage{ngerman,american}{babel}   % change this to your language(s)
% Spanish languages need extra options in order to work with this template
%\PassOptionsToPackage{spanish,es-lcroman}{babel}
\usepackage{babel}
\usepackage{subcaption}

\usepackage[utf8]{inputenc}

\usepackage{algorithm} 
\usepackage{algpseudocode}

\usepackage[dvipsnames]{xcolor} % Pro­vides easy driver-in­de­pen­dent ac­cess to sev­eral kinds of colors
\definecolor{tsKeyword}{RGB}{0,0,255}
\definecolor{tsString}{RGB}{163,21,21}
\definecolor{tsComment}{RGB}{0,128,0}
\definecolor{tsType}{RGB}{43,145,175}
\definecolor{tsNumber}{RGB}{128,0,128}


\usepackage{listingsutf8}
\lstdefinelanguage{json}{
    basicstyle=\ttfamily,
    morestring=[b]",
    morecomment=[l]{//},
    morecomment=[s]{/*}{*/},
    stringstyle=\color{violet},
    commentstyle=\color{green},
    showstringspaces=false
}

\lstdefinelanguage{typescript}{
    keywords={
        break, case, catch, class, const, continue, debugger, default, delete, do, else,
        enum, export, extends, false, finally, for, function, if, import, in, instanceof,
        new, null, return, super, switch, this, throw, true, try, typeof, var, void, while, with, let, await, async, static, as
    },
    keywordstyle=\color{tsKeyword}\bfseries,
    ndkeywords={
        boolean, number, string, symbol, any, never, unknown, undefined, null, object,
        interface, type, readonly, Record, Partial, Pick, Omit
    },
    ndkeywordstyle=\color{tsType}\bfseries,
    identifierstyle=\color{black},
    sensitive=true,
    comment=[l]{//},
    morecomment=[s]{/*}{*/},
    commentstyle=\color{tsComment}\itshape,
    stringstyle=\color{tsString},
    morestring=[b]',
    morestring=[b]",
    morestring=[b]`,
    numbers=left,
    numberstyle=\tiny\color{gray},
    stepnumber=1,
    numbersep=10pt,
    backgroundcolor=\color{white},
    showspaces=false,
    showstringspaces=false,
    showtabs=false,
    tabsize=2,
    breaklines=true,
    breakatwhitespace=true,
    basicstyle=\ttfamily\small,
    captionpos=b
}

\lstset{
  language=typescript,
  inputencoding=utf8,
  extendedchars=true,
  literate={ä}{{\"a}}1 {ö}{{\"o}}1 {ü}{{\"u}}1 {ß}{{\ss}}1
           {Ä}{{\"A}}1 {Ö}{{\"O}}1 {Ü}{{\"U}}1,
  basicstyle=\ttfamily\small,
  breaklines=true
}
